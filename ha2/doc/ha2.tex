\documentclass[a4paper]{article}

\usepackage[utf8]{inputenc}

\usepackage{url}
\usepackage[]{hyperref}

\usepackage{caption}

\usepackage{listings}

\usepackage{color}

\usepackage{pythonhighlight}

% *** GRAPHICS RELATED PACKAGES ***
%\usepackage[pdftex]{graphicx}
\usepackage{graphicx}
%\usepackage[dvips]{graphicx}
% to place figures on a fixed position
\usepackage{float}

\usepackage[margin=1in]{geometry}

\title{Hyperbolic Geometry of Complex Networks – Queuing Theory I. Home Assignment II.}
\author{Ferenc Nandor Janky - OA8AT9}
\date{}


\begin{document}

\maketitle

\tableofcontents

\section{Introduction}

The task was to study section I, II, III and IV A of~\cite{HyperbolicGeoNetworks} and in any programming language/tool generate a network according to the model described in the paper 
(in Section IV.A.) with the following parameters: N=5000 (the number of nodes), R=14 (the radius of the disk on which the nodes are uniformly distributed). Then calculate numerically the empirical average degree, and plot on a log-log scale the empirical degree distribution of this generated network.

\section{Implementation}
The implementation of the graph generation and analysis has been done in Python language. The software is available on \url{https://github.com/fecjanky/QT_home_assignment/blob/master/ha2/toki2.py}.
The graph generation has been implemented as described in ~\cite{HyperbolicGeoNetworks}. The node density on the disc with radius \emph{R} along the radial polar coordinates followed an exponential distribution $ \rho~(r) \simeq e^r $~.
The connection probability was analogous to the hyperbolic distance between nodes on the disc given by: $ p(x) = \Theta(R - x) $
For the implementation the following Python libraries have been utilized:
\begin{itemize}
\item \verb!matplotlib! , for creating the representation of the generated graph in polar coordinates and for plotting the degree distribution
\item \verb!networkx! , for graph analysis
\end{itemize}

The  generator function of the points is show in listing~\ref{lst:python}. There were some offset between the simulated and theoretical results (see Section~\ref{sect:metrics}) and it could have been caused by a bias in the generation of random nodes.

\newpage

\begin{lstlisting}[style=mypython,caption={The function used for generating random nodes},label={lst:python}]
    def lte(a, b):
        return math.isclose(a, b) or a < b
	
    # use rejection sampling to generate points with a given distribution
    def generate_points(self, distribution=None):
        points = []
        if distribution is None:
            distribution = lambda r: math.sinh(r) / (math.cosh(self.radius) - 1)

        for i in range(0, self.nodecount):
            azimuth = random.uniform(0, 2 * math.pi)
            d_point = (random.uniform(0, self.radius), random.uniform(0, distribution(self.radius)))
            while True:
                d_accept = distribution(d_point[0])
                if lte(d_point[1], d_accept):
                    break
                d_point = (random.uniform(0, self.radius), random.uniform(0, distribution(self.radius)))
            points.append(PolarPoint(radius=d_point[0], azimuth=azimuth))
        return points
\end{lstlisting}

\section{Results}
\subsection{Theoretical metrics}
To calculate the theoretical average degree the following equations were used from \cite{HyperbolicGeoNetworks}:

\begin{equation}
N = \nu~e^{R/2} \rightarrow \nu \simeq 4.559 ,if\;R=14\;and\;N=5000
\end{equation}
\begin{equation}
R = 2 \ln[8 N / (\pi \overline{k})] \rightarrow \overline{k} \simeq 11.61, if\;R=14\;and\;N=5000
\end{equation}

\subsection{Empirical metrics}\label{sect:metrics}

The polar plot of the generated network can be seen on Figure~\ref{fig:graph}. The empirical average degree calculated for the generated graph was:
\begin{equation}
\overline{k}_{sim} =  11.8084
\end{equation}

The relative error between the average obtained from the generated one and the theoretical average was:
\begin{equation}
\epsilon = \frac{\vert \overline{k} - \overline{k}_{sim} \vert}{\overline{k} } * 100 \% = 1.71 \%
\end{equation}

\begin{figure}[H]
    \centering
    \includegraphics[width=0.9\textwidth]{figures/result_graph.png}
    \caption{The polar plot of the generated graph according to the rules in \cite{HyperbolicGeoNetworks}}
    \label{fig:graph}
\end{figure}


Figure~\ref{fig:graph_stats} shows the empirical degree distribution of the generated network on a log-log scale. It resembles the expected Poissonian distribution.
 
\begin{figure}[H]
    \centering
    \includegraphics[width=0.9\textwidth]{figures/result_graph_stats.png}
    \caption{The plot on a log-log scale of the empirical degree distribution of the generated network}
    \label{fig:graph_stats}
\end{figure}


The average degree as a function of radius from the center on the disc is illustrated on Figure~\ref{fig:graph_radius} alongside with the theoretical curve. The tangent of the empirical curve is similar however there was a constant offset between them that might have been cause by the bias in the random point generation.

\begin{figure}[H]
    \centering
    \includegraphics[width=0.9\textwidth]{figures/result_graph_radial_stats.png}
    \caption{The plot of the average degree as a function of radial distance of the generated network}
    \label{fig:graph_radius}
\end{figure}




\section{Conclusion}

As a part of this homework a random network has been generated using  hyperbolic geometry based on \cite{HyperbolicGeoNetworks} to study the
structure and function of complex network in purely geometric terms. The edge probability is analogous to the hyperbolic distance between two nodes and if that distance is below a threshold between 
any of two nodes an edge is present between them resulting in similar structure as it would have been created by specifying and edge probability for the random graph and generating edges between them based on that.
The simulation results were resembling the theoretical results and also the ones presented in \cite{HyperbolicGeoNetworks} with around 2\% relative error.


\bibliographystyle{unsrt}
\bibliography{references}

\end{document}